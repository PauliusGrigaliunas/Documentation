\documentclass{VUMIFPSkursinis}
\usepackage{algorithmicx}
\usepackage{algorithm}
\usepackage{algpseudocode}
\usepackage{amsfonts}
\usepackage{amsmath}
\usepackage{bm}
\usepackage{caption}
\usepackage{color}
\usepackage{float}
\usepackage{graphicx}
\usepackage{listings}
\usepackage{subfig}
\usepackage{wrapfig}
\usepackage{pdflscape} %Keep it to pdflscape or I can't rotate my diagram (K.S.)
\usepackage{longtable}
\usepackage[table]{xcolor}
\usepackage{multirow}
\usepackage[usestackEOL]{stackengine}
\usepackage{longtable}

\usepackage{enumitem}
%PAKEISTA, tarpai tarp sąrašo elementų
\setitemize{noitemsep,topsep=0pt,parsep=0pt,partopsep=0pt}
\setenumerate{noitemsep,topsep=0pt,parsep=0pt,partopsep=0pt}

% Titulinio aprašas
\university{Vilniaus universitetas}
\faculty{Matematikos ir informatikos fakultetas}
\department{Programų sistemų katedra}
\papertype{Programų sistemų inžinerijos I laboratorinis darbas Nr. 2}
\title{Reikalavimų specifikacija}
\titleineng{}
\status{2 kurso 5 grupės studentai}
\author{Paulius Grigaliūnas}
\secondauthor{Karolis Staskevičius}
\thirdauthor{Modestas Dulevičius}
\fourthauthor{Albert Jurkoit}
\fifthauthor{Šarūnas Kazimieras Buteikis}
     

% \secondauthor{Vardonis Pavardonis}   % Pridėti antrą autorių
\supervisor{dr. Vytautas Valaitis}
\date{Vilnius – \the\year}

% Nustatymai
% \setmainfont{Palemonas}   % Pakeisti teksto šriftą į Palemonas (turi būti įdiegtas sistemoje)

\begin{document}
% PAKEISTA	
\maketitle
\cleardoublepage\pagenumbering{arabic}
\setcounter{page}{2}


%ANOTACIJA

\sectionnonum{ANOTACIJA}
{\bfseries Darbo tikslas:} Reikalavimų anotacija Plz kažkas parašykite :). 
\newline
\newline
\newline

%DARBO ATLIKO
\noindent
{\bfseries Darbą atliko:}
\newline
\newline
\newline
Paulius Grigaliūnas
\newline
paulius.grigaliunas.pg@gmail.com
\newline
\newline
\newline
Karolis Staskevičius
\newline
satelistas@gmail.com
\newline
\newline
\newline
Modestas Dulevičius
\newline
modux9@gmail.com
\newline
\newline
\newline
Albert Jurkoit
\newline
albert.jurkoit@mif.stud.vu.lt
\newline
\newline
\newline
Šarūnas Kazimieras Buteikis
\newline
sarunas.kazimieras.buteikis@gmail.com

%TURINYS(TOC)
\tableofcontents

%ĮVADAS
\sectionnonum{Įvadas}

Čionais bus mūsų įvadas. Vėliau jį užpildysime.
\newline

%FUNKCINIAI REIKALAVIMAI

\section{Vartotojo sąsaja}

\subsection{Formuluojamos užduotys}
\subsubsection{Vartotojo užduotys}
\begin{center}
	\begin{table}[H]
	\begin{tabular}{|p{2cm}|p{16cm}|}
	\hline
	    \rowcolor{lightgray}
		\multicolumn{2}{|c|}{Vartotojo užduotys}\\
		
	\hline
		\multicolumn{1}{|c|}{{\bfseries Užduotis}}&
		\multicolumn{1}{|c|}{{\bfseries Formulavimo būdas}}\\		
	\hline 	
		\multicolumn{1}{|c|}{Autentifikavimo užduotys}&
		\multicolumn{1}{|p{11,6cm}|}{
			\begin{enumerate}
				\item Registruotis sistemoje
				(privalomi duomenys:vardas, pavardė, el. paštas, telefono numeris, slaptažodis).
				\item Prisijungti prie sistemos (privalomi duomenys: el. paštas, slaptažodis).
				\item “Premium” prisijungimas (aktyvuojamas vartotojui, kuris nupirko tam tikrą paslaugą) leidžiantis prisijungti prie sistemos vienu mygtuko paspaudimu.
			\end{enumerate}}\\
	
	\hline 	
		\multicolumn{1}{|c|}{Pagalbinės užduotys}&
		\multicolumn{1}{|p{11,6cm}|}{
			\begin{enumerate}
				\item Peržiūrėti visas sistemoje registruotas kavinės, informaciją apie jas.
				\item Ieškoti kavinę pagal vardą.
				\item Ieškoti kavinę pagal vietą.
				\item Peržiūrėti pasirinktos kavinės detaliąją informaciją (Vardas, adresas, staliukų skaičius, darbo grafikas).
				\item Rezervuoti staliuką pasirinktoje kavinėje.
			\end{enumerate}}\\
	
	\hline 	 	
	\end{tabular}
	\end{table}

\end{center}

\subsubsection{Kavinės savininko užduotys}
\begin{center}
	\begin{table}[H]
	\begin{tabular}{|p{2cm}|p{16cm}|}
	\hline
	    \rowcolor{lightgray}
		\multicolumn{2}{|c|}{Kavinės savininko užduotys}\\
		
	\hline
		\multicolumn{1}{|c|}{{\bfseries Užduotis}}&
		\multicolumn{1}{|c|}{{\bfseries Formulavimo būdas}}\\		
	\hline 	
		\multicolumn{1}{|c|}{Autentifikavimo užduotys}&
		\multicolumn{1}{|p{11,6cm}|}{
			\begin{enumerate}
				\item Registruotis sistemoje (privalomi duomenys:vardas, pavardė, el. paštas, telefono numeris, slaptažodis).
				\item Prisijungti prie sistemos (privalomi duomenys: el. paštas, slaptažodis).
			\end{enumerate}}\\
	\hline
	    \rowcolor{lightgray}
		\multicolumn{1}{|c|}{Pagalbinės užduotys}\\
		

	\hline 	
		\multicolumn{1}{|c|}{Autentifikavimo užduotys}&
		\multicolumn{1}{|p{11,6cm}|}{
			\begin{enumerate}
				\item Pridėti kavinę (privalomi duomenys: kavinės vardas, adresas, staliukų skaičius, darbo grafikas. Papildomi: kavinės telefono numeris).
				\item Peržiūrėti sistemoje registruotų kavinių sąrašą.
				\item Atnaujinti informaciją apie savo registruotą kavinę.
			\end{enumerate}}\\
	
	\hline 	 	
	\end{tabular}
	\end{table}

\end{center}

\subsubsection{Užduočių formulavimo kalbos reikalavimai}
\begin{center}

	\begin{longtable}{|p{2cm}|p{16cm}|}
	\hline
	    \rowcolor{lightgray}
		\multicolumn{2}{|c|}{Užduočių formulavimo kalbos reikalavimai}\\
		
	\hline
		\multicolumn{1}{|c|}{{\bfseries Užduotis}}&
		\multicolumn{1}{|c|}{{\bfseries Formulavimo būdas}}\\		
	\hline 	
		\multicolumn{1}{|c|}{Registruotis}&
		\multicolumn{1}{|p{8,6cm}|}{
			\begin{enumerate}
				\item Duomenų įvedimo laukeliai.
				\item Registracijos mygtukas.
			\end{enumerate}}\\
	
	\hline 	
		\multicolumn{1}{|c|}{Prisijungti}&
		\multicolumn{1}{|p{8,6cm}|}{
			\begin{enumerate}
				\item Piktrograma. 
				\item Duomenų įvedimo laukeliai.
				\item Prisijungimo  mygtukas.
			\end{enumerate}}\\
	
	\hline 	
		\multicolumn{1}{|c|}{Premium prisijungimas}&
		\multicolumn{1}{|p{8,6cm}|}{
			\begin{enumerate}
				\item Premium prisijungimo mygtukas.
			\end{enumerate}}\\
	
	\hline 	
		\multicolumn{1}{|c|}{Kavinės pridėjimas}&
		\multicolumn{1}{|p{8,6cm}|}{
			\begin{enumerate}
				\item Duomenų įvedimo laukeliai.
				\item Pridėjimo mygtukas.
			\end{enumerate}}\\
	
	\hline 	
		\multicolumn{1}{|c|}{Išeiti}&
		\multicolumn{1}{|p{8,6cm}|}{
			\begin{enumerate}
				\item Išėjimo mygtukas.
			\end{enumerate}}\\
	
	\hline	
		
		\multicolumn{1}{|c|}{Kavinės paieška}&
		\multicolumn{1}{|p{10,2cm}|}{
			\begin{enumerate}
				\item Lentelė su sistemoje registruotomis kavinėmis.
				\item Kavinės rezervacijos mygtukas.
				\item Paieškos duomenų įvedimo laukelis.
				\item Paieškos pagal vardą mygtukas.
				\item Paieškos pagal vietą mygtukas.
				\item Parodyti daugiau informacijos mygtukas.
			\end{enumerate}}\\
	
	\hline 	
		\multicolumn{1}{|c|}{Kavinės rezervacija}&
		\multicolumn{1}{|p{10,2cm}|}{
			\begin{enumerate}
				\item Datos (metai, mėnuo, diena) nustatymo laukelis
				\item Laiko (valandos, minutės) nustatymo laukelis.
				\item Lango uždarymo mygtukas.
				\item Rezervacijos mygtukas.
			\end{enumerate}}\\
	
	\hline 	
		\multicolumn{1}{|c|}{Kavinės informacijos atnaujinimas}&
		\multicolumn{1}{|p{8,6cm}|}{
			\begin{enumerate}
				\item Duomenų įvedimo laukeliai.
				\item Patvirtinimo mygtukas.
			\end{enumerate}}\\
	
	\hline
		
	\end{longtable}

\end{center}

\subsubsection{Interfeiso darnos ir standartizavimo reikalavimai}
\begin{center}
	\begin{table}[H]
	\begin{tabular}{|p{16cm}|p{2cm}|}
	\hline
	    \rowcolor{lightgray}
		\multicolumn{1}{|c|}{ {\bfseries Užduotis}}&
		\multicolumn{1}{|c|}{{\bfseries Svarba}}\\		
	\hline
		\multicolumn{1}{|p{14,5cm}|}{Pagrindinis meniu visuomet pasiekiamas paspaudus ant programėlės logotipo esančio kairiajame viršutiniame kampe.}& 
		\multicolumn{1}{|p{1.5cm}|}{PAGEIDAUTINA}\\
	\hline
		\multicolumn{1}{|p{14,5cm}|}{Languose dominuoja „Royal Jewels“ spalvų paletė.}& 
		\multicolumn{1}{|p{1.5cm}|}{PAGEIDAUTINA}\\
	\hline
		\multicolumn{1}{|p{14,5cm}|}{Tekstui naudojami ‚Times New Roman“ šriftas}& 
		\multicolumn{1}{|p{1.5cm}|}{PAGEIDAUTINA}\\
	\hline  	 	
	
	\end{tabular}
	\end{table}

\end{center}

\subsubsection{Pranešimų formulavimo reikalavimai}
\begin{center}
	\begin{table}[H]
	\begin{tabular}{|p{16cm}|p{2cm}|}
	\hline
	    \rowcolor{lightgray}
		\multicolumn{1}{|c|}{ {\bfseries Užduotis}}&
		\multicolumn{1}{|c|}{{\bfseries Svarba}}\\		
	\hline
	
		\multicolumn{1}{|p{14,5cm}|}{Pranešimų tekstas turi būti parašytas laikantis gramatikos ir skyrybos taisyklių.}& 
		\multicolumn{1}{|c|}{BŪTINA}\\
	\hline
	
		\multicolumn{1}{|p{14,5cm}|}{Pranešime vartojami tik interfeiso naudotojams žinomi terminai, vengiama žargonų ir svetimybių, tačiau leistina vartoti dalykinės srities metaforas.}& 
		\multicolumn{1}{|c|}{BŪTINA}\\
	\hline
	
		\multicolumn{1}{|p{14,5cm}|}{Informacinio tipo pranešimai aprašys domenų įvedimo kriterijus.}& 
		\multicolumn{1}{|c|}{BŪTINA}\\
	\hline
	
		\multicolumn{1}{|p{14,5cm}|}{Pranešimų langai įspės apie blogai įvestus duomenis, pavyzdžiui, per trumpą įvestą slaptažodį arba jau užregistruotą el. paštą.}& 
		\multicolumn{1}{|c|}{BŪTINA}\\
	\hline
	
		\multicolumn{1}{|p{14,5cm}|}{Pranešimas turi pateikti informaciją apie sėkmingai atliktą veiksmą arba informuoti apie klaidą.}& 
		\multicolumn{1}{|c|}{BŪTINA}\\
	\hline
	
		\multicolumn{1}{|p{14,5cm}|}{Pranešimo tekstas turi būti suprantamas vienareikšmiškai.}& 
		\multicolumn{1}{|c|}{BŪTINA}\\
	\hline
	
		\multicolumn{1}{|p{14,5cm}|}{Pranešimas apie klaidą turi būti detalus.}& 
		\multicolumn{1}{|p{1.5cm}|}{PAGEIDAUTINA}\\
	\hline
	
		\multicolumn{1}{|p{14,5cm}|}{Informacinis pranešimas turi būti žalios arba mėlynos spalvos, o klaidos pranešimas - raudonos.}& 
		\multicolumn{1}{|p{1.5cm}|}{PAGEIDAUTINA}\\
	\hline
	
		\multicolumn{1}{|p{14,5cm}|}{Pradinis pranešimo langas negali užimti daugiau nei 50\% ekrano pločio ir aukščio.}& 
		\multicolumn{1}{|p{1.5cm}|}{PAGEIDAUTINA}\\
	\hline
	
		\multicolumn{1}{|p{14,5cm}|}{Pranešimo teksto ilgis negali viršyti 100 simbolių limito.}& 
		\multicolumn{1}{|p{1.5cm}|}{PAGEIDAUTINA}\\
	\hline
	
		\multicolumn{1}{|p{14,5cm}|}{Pranešimas turi turėti antraštę.}& 
		\multicolumn{1}{|p{1.5cm}|}{PAGEIDAUTINA}\\
	\hline
	
		\multicolumn{1}{|p{14,5cm}|}{Turi būti galimybė išjungti pranešimo langą.}& 
		\multicolumn{1}{|c|}{BŪTINA}\\
	\hline	 	
	
	\end{tabular}
	\end{table}

\end{center}

\pagebreak

\section{Funkciniai reikalavimai}
%PRADEDAME RASYTI LENTELES WOHOO

Šiame skyriuje bus nurodomi aplikacijos funkciniai reikalavimai ir jų įgyvendinimo svarba.

\subsection{Aplikacijos langai}

 Šiame poskyryje pateikti funkciniai reikalavimai, susiję su aplikacijos langais (žr.~\ref{table:AplikacijosLangai} lentelę).
 %APLIKACIJOS LANGAI
\begin{center}
	\begin{table}[H]
	\begin{tabular}{|p{2cm}|p{12,5cm}|p{3,5cm}|}
	\hline
	    \rowcolor{lightgray}
		\multicolumn{3}{|c|}{Aplikacijos langai}\\
		
	\hline
		\multicolumn{1}{|c|}{{\bfseries Kodas}}&
		\multicolumn{1}{|c|}{{\bfseries Reikalavimas}}&
		\multicolumn{1}{|c|}{{\bfseries Svarba}}\\

	\hline
		\multicolumn{1}{|c|}{FR 1.1} &
		Iš visų aplikacijos langų galima grįžti į prisijungimo langą &
		\multicolumn{1}{|c|}{BŪTINA}\\
	\hline
		\multicolumn{1}{|c|}{FR 1.2} &
		{Visi langai po prisijungimo turi turėti galimybę parodyti vartotojo \newline prisijungimo vardą/paštą.}&
		\multicolumn{1}{|c|}{BŪTINA}\\
	\hline
	
		\multicolumn{1}{|c|}{FR 1.3}&
		{Jeigu vartotojas turi rezervavęs staliuką - iš visų langų, išskyrus \newline prisijungimo, galima patekti į langą rodantį rezervacijos informaciją.}&
		\multicolumn{1}{|c|}{BŪTINA}\\
	\hline
		\multicolumn{1}{|c|}{FR 1.4}&
		{Iš prisijungimo lango galima patekti į registracijos ir meniu langą.}&
		\multicolumn{1}{|c|}{BŪTINA}\\
	\hline
		\multicolumn{1}{|c|}{FR 1.5}&
		{Iš meniu lango galima patekti į:\newline
		\begin{enumerate}
			\item Jei prisijungia kavinės administratorius:
				\begin{enumerate}
					\item “Keisti kavinės informaciją” langą.
					\item “Keisti kavinės išplanavimą” langą.
				\end{enumerate}
			\item Jei prisijungia kavinės darbuotojas:
				\begin{enumerate}
					\item “Lokali rezervacija” langą.
				\end{enumerate}
			\item Jei prisijungia klientas:
				\begin{enumerate}
					\item “Kavinės paieška” langas.
				\end{enumerate}
			\item Visi prisijungę vartotojai turi galimybę patekti į “Keisti vartotojo duomenis” langą.
			\item Visi langai turi turėti galimybę grįžti į ankstesnį langą.
		\end{enumerate}
		}&
		\multicolumn{1}{|c|}{BŪTINA}\\		
	\hline
	
	\end{tabular}
	\caption{Aplikacijos langų funkciniai reikalavimai.}
	\label{table:AplikacijosLangai}
	\end{table}
	
\end{center}

\pagebreak

\subsection{Vartotojo registracija}

Šiame poskyryje pateikti funkciniai reikalavimai, susiję su vartotojo registracija (žr.~\ref{table:VartotojoRegistracija} lentelę).
%Vartotojo prisijungimas
\begin{center}

	\begin{table}[H]
	\begin{tabular}{|p{2cm}|p{12,5cm}|p{3,5cm}|}
	\hline
	    \rowcolor{lightgray}
		\multicolumn{3}{|c|}{Vartotojo registracija}\\
		
	\hline
		\multicolumn{1}{|c|}{{\bfseries Kodas}}&
		\multicolumn{1}{|c|}{{\bfseries Reikalavimas}}&
		\multicolumn{1}{|c|}{{\bfseries Svarba}}\\

	\hline
	\multicolumn{1}{|c|}{FR 2.1}&
	{Vartotojas, norintis registruotis prie aplikacijos, turi nurodyti:
		\begin{itemize}
			\item Vardą.
			\item Pavardę.
			\item Slaptažodį.
			\item Telefono numerį (neprivaloma).
			\item Elektroninį paštą.
		\end{itemize}}&		
	\multicolumn{1}{|c|}{BŪTINA}\\
	\hline
	
		\multicolumn{1}{|c|}{FR 2.2}&
		{Informuoti vartotoją apie sėkmingą registraciją}&
		\multicolumn{1}{|c|}{BŪTINA}\\	
	\hline
		\multicolumn{1}{|c|}{FR 2.3}&
		{Vartotojo duomenys išsaugomi duomenų bazėje}&
		\multicolumn{1}{|c|}{BŪTINA}\\	
	\hline	
		\multicolumn{1}{|c|}{FR 2.4}&
		{Tam, kad įvestas vartotojo el. pašto adresas būtų tinkamos formos, jis turi būti sudarytas iš abonento vardo, „@“ simbolio bei domeno adreso.}&
		\multicolumn{1}{|c|}{BŪTINA}\\	
	\hline
		\multicolumn{1}{|c|}{FR 2.5}&
		{Slaptažodis privalo būti sudarytas bent iš 8 simbolių, turi bent vieną skaitmenį.
}&
		\multicolumn{1}{|c|}{BŪTINA}\\		
	\hline
		\multicolumn{1}{|c|}{FR 2.6}&
		{Duomenų bazėje išsaugomi vartotojo duomenys turi būti string tipo }&
		\multicolumn{1}{|p{1.5cm}|}{PAGEIDAUTINA}\\		
	\hline
	
	\end{tabular}
	\caption{Vartotojo registracijos funkciniai reikalavimai}
	\label{table:VartotojoRegistracija}
	\end{table}

\end{center}

\pagebreak

\subsection{Vartotojo prisijungimas}
Šiame poskyryje pateikti reikalavimai, susiję su vartotojo prisijungimu (žr.~\ref{table:VartotojoPrisijungimas} lentelę).

\begin{center}
	\begin{table}[H]
	\begin{tabular}{|p{2cm}|p{12,5cm}|p{3,5cm}|}
	\hline
	    \rowcolor{lightgray}
		\multicolumn{3}{|c|}{Vartotojo prisijungimas}\\
		
	\hline
		\multicolumn{1}{|c|}{{\bfseries Kodas}}&
		\multicolumn{1}{|c|}{{\bfseries Reikalavimas}}&
		\multicolumn{1}{|c|}{{\bfseries Svarba}}\\

	\hline
	
		\multicolumn{1}{|c|}{FR 3.1}&
		{Vartotojas, norintis prisijungti prie aplikacijos turi nurodyti:
			\begin{itemize}
				\item Elektroninio pašto adresas.
				\item Slaptažodis.
			\end{itemize}
		}&
		\multicolumn{1}{|c|}{BŪTINA}\\	
		
	\hline	
		\multicolumn{1}{|c|}{FR 3.2}&
		{Vartotojas prijungiamas prie sistemos ir nukreipiamas į meniu langą kartu su papildomais vartotojo duomenimis.}&
		\multicolumn{1}{|c|}{BŪTINA}\\
		
	\hline	
		\multicolumn{1}{|c|}{FR 3.3}&
		{Jei prisijungimo duomenys neteisingi, vartotojas informuojamas pranešimu.}&
		\multicolumn{1}{|c|}{BŪTINA}\\		
		
	\hline
		\multicolumn{1}{|c|}{FR 3.4}&
		{Vartotojo įvesti duomenys (slaptažodis bei el. pašto adresas) turi būti sutikrinti su duomenimis, esančiais duomenų bazėje.}&
		\multicolumn{1}{|c|}{BŪTINA}\\
		
	\hline	
		\multicolumn{1}{|c|}{FR 3.5}&
		{Jei prisijungimo duomenys neteisingi, vartotojas informuojamas pranešimu.}&
		\multicolumn{1}{|c|}{BŪTINA}\\
	\hline
	
	
	
	\end{tabular}	
	\caption{Vartotojo prisijungimo funkciniai reikalavimai}
	\label{table:VartotojoPrisijungimas}		
	\end{table}

\end{center}
\pagebreak

\subsection{Kavinės pasirinkimas}

Šiame posyryje pateikti reikalavimai, susiję su kavinės pasirinkimu (žr.~\ref{table:KavinėsPasirinkimas} lentelę ).

\begin{center}
	\begin{table}[H]
	\begin{tabular}{|p{2cm}|p{12,5cm}|p{3,5cm}|}
	
	\hline
	    \rowcolor{lightgray}
		\multicolumn{3}{|c|}{Kavinės pasirinkimas}\\
		
	\hline
		\multicolumn{1}{|c|}{{\bfseries Kodas}}&
		\multicolumn{1}{|c|}{{\bfseries Reikalavimas}}&
		\multicolumn{1}{|c|}{{\bfseries Svarba}}\\

	\hline
	
		\multicolumn{1}{|c|}{FR 4.1}&
		{Klientui pateikiamas sąrašas arčiausiai jo esančių kavinių kartu su jų duomenimis (ne visais):
		\begin{itemize}
			\item Pavadinimas.
			\item Adresas.
			\item Reitingas.
		\end{itemize}}&
		\multicolumn{1}{|c|}{BŪTINA}\\	
		
	\hline
	
		\multicolumn{1}{|c|}{FR 4.2}&
		{Klientas gali paspausti ir pasirinkti vieną iš kavinių.}&
		\multicolumn{1}{|c|}{BŪTINA}\\
		
	\hline
	
		\multicolumn{1}{|c|}{FR 4.4}&
		{Vartotojui pasirinkus kavinę, parodoma visa jos informacija:
		\begin{itemize}
			\item Pavadinimas.
			\item Adresas.
			\item Vietų skaičius.
			\item Laivų vietų skaičius.
			\item Reitingas.
			\item Darbo valandos.
		\end{itemize}}&
		\multicolumn{1}{|c|}{BŪTINA}\\
		
	\hline
	
		\multicolumn{1}{|c|}{FR 4.5}&
		{Vartotojas, pasirinkęs kavinę (turinčią laisvų vietų), gali užsisakyti staliuką.}&
		\multicolumn{1}{|c|}{BŪTINA}\\
	\hline
	
		\multicolumn{1}{|c|}{FR 4.6}&
		{Jeigu vartotojas nori užsisakyti staliuką, jis yra nukreipiamas į staliuko užsakymo langą}&
		\multicolumn{1}{|c|}{BŪTINA}\\				
	\hline
	
	\end{tabular}		
	\caption{Kavinės pasirinkimo funkciniai reikalavimai}
	\label{table:KavinėsPasirinkimas}
	\end{table}


\end{center}

\pagebreak


\subsection{Staliuko užsakymas}
Šiame poskyryje pateikti reikalavimai, susiję su staliuko užsakymais(žr.~\ref{table:StaliukoUžsakymas} lentelę).
\begin{center}
	\begin{table}[H]
	\begin{tabular}{|p{2cm}|p{12,5cm}|p{3,5cm}|}
	
	\hline
	    \rowcolor{lightgray}
		\multicolumn{3}{|c|}{Staliuko užsakymas}\\
		
	\hline
		\multicolumn{1}{|c|}{{\bfseries Kodas}}&
		\multicolumn{1}{|c|}{{\bfseries Reikalavimas}}&
		\multicolumn{1}{|c|}{{\bfseries Svarba}}\\

	\hline
	
		\multicolumn{1}{|c|}{FR 5.1}&
		{Vartotojui pateikiamas kavinės išplanavimas su staliukais.}&
		\multicolumn{1}{|c|}{BŪTINA}\\				
	\hline
	
		\multicolumn{1}{|c|}{FR 5.2}&
		{Vartotojui pasirinkęs vieną iš laisvų staliukų nukreipiamas į langą, kuriame prašo įvesti rezervacijos datą ir laiką.}&
		\multicolumn{1}{|c|}{BŪTINA}\\				
	\hline
	
		\multicolumn{1}{|c|}{FR 5.3}&
		{Jei vartotojas nėra Premium ir kavinėje likęs tik vienas laisvas staliukas - klientas neturi galimybės jo rezervuoti, gauna pranešimą, kad neturi Premium privilegijų.}&
		\multicolumn{1}{|c|}{BŪTINA}\\				
	\hline
	
		\multicolumn{1}{|c|}{FR 5.4}&
		{Vartotojas norėdamas rezervuoti vietą restorane privalo nurodyti:
			\begin{itemize}
				\item Rezervavimo datą.
				\item Rezervavimo laiką.
			\end{itemize}}&
			\multicolumn{1}{|c|}{BŪTINA}\\				
	\hline
	
		\multicolumn{1}{|c|}{FR 5.3}&
		{Duomenų bazėje patikrinus ar staliukas vis dar laisvas į ją siunčiamas rezervacijos data ir laikas, kartu su užsisakiusio vartotojo vardu, paštu ir telefono numeriu (jei yra įvestas)}&
		\multicolumn{1}{|c|}{BŪTINA}\\				
	\hline
	
			
	
	\end{tabular}		
	\caption{Staliuko užsakymo funkciniai reikalavimai.}
	\label{table:StaliukoUžsakymas}
	\end{table}


\end{center}

\pagebreak

\subsection{Rezervacijų peržiūra}
Šiame poskyryje pateikti reikalavimai, susiję su rezervacijų priežiūra (žr.~\ref{tabel:RezervacijosPeržiūra} lentelę).
\begin{center}
	\begin{table}[H]
	\begin{tabular}{|p{2cm}|p{12,5cm}|p{3,5cm}|}
	
	\hline
	    \rowcolor{lightgray}
		\multicolumn{3}{|c|}{Rezervacijų priežiūra}\\
		
	\hline
		\multicolumn{1}{|c|}{{\bfseries Kodas}}&
		\multicolumn{1}{|c|}{{\bfseries Reikalavimas}}&
		\multicolumn{1}{|c|}{{\bfseries Svarba}}\\

	\hline
	
		\multicolumn{1}{|c|}{FR 6.1}&
		{Vartotojas mato savo rezervacijų sąrašą.}&
		\multicolumn{1}{|c|}{BŪTINA}\\				
	\hline
	
		\multicolumn{1}{|c|}{FR 6.2}&
		{Vartotojas mato savo rezervacijų sąrašą.}&
		\multicolumn{1}{|c|}{BŪTINA}\\				
	\hline
	
		\multicolumn{1}{|c|}{FR 6.3}&
		{Vykdant rezervacijos atšaukimą į duomenų bazę siunčiama užklausa atlaisvinti staliuką. Rezervacijos įrašas pašalinamas iš kliento rezervacijos sąrašo.}&
		\multicolumn{1}{|c|}{BŪTINA}\\				
	\hline
	
		\multicolumn{1}{|c|}{FR 6.4}&
		{Vartotojas gali peržiūrėti kavinės rezervaciją. Peržiūrint kavinės informaciją pateikiamas kavinės duomenų langas.}&
		\multicolumn{1}{|c|}{BŪTINA}\\				
	\hline		
	
	\end{tabular}		
	\caption{Rezervacijos peržiūros funkciniai reikalavimai.}
	\label{tabel:RezervacijosPeržiūra}
	\end{table}


\end{center}



\subsection{Kavinės įvertinimas}
Šiame poskyryje pateikiami reikalavimai, susiję su kavinės įvertinimu (žr. ~\ref{table:KavinėsĮvertinimas} lentelę).
\begin{center}
	\begin{table}[H]
	\begin{tabular}{|p{2cm}|p{12,5cm}|p{3,5cm}|}
	
	\hline
	    \rowcolor{lightgray}
		\multicolumn{3}{|c|}{Kavinės įvertinimas}\\
		
	\hline
		\multicolumn{1}{|c|}{{\bfseries Kodas}}&
		\multicolumn{1}{|c|}{{\bfseries Reikalavimas}}&
		\multicolumn{1}{|c|}{{\bfseries Svarba}}\\

	\hline
	
		\multicolumn{1}{|c|}{FR 7.1}&
		{Kavinės valdytojui pažymėjus, kad prieš tai vartotojo rezervuotas staliukas atsilaisvino, klientas gauna langą, kuriame prašoma įvertinti kavinę.}&
		\multicolumn{1}{|c|}{BŪTINA}\\				
	\hline
	
		\multicolumn{1}{|c|}{FR 7.2}&
		{Vartotojas gali atsisakyti įvertinti kavinę.}&
		\multicolumn{1}{|c|}{BŪTINA}\\				
	\hline
	
		\multicolumn{1}{|c|}{FR 7.3}&
		{Atsisakius įvertinti kavinę, automatiškai išeinama iš kavinės įvertinimo lango.}&
		\multicolumn{1}{|c|}{BŪTINA}\\				
	\hline
	
		\multicolumn{1}{|c|}{FR 7.4}&
		{Vartotojas gali įvertinti kavinę nuo 1 iki 5}&
		\multicolumn{1}{|c|}{BŪTINA}\\				
	\hline
	
		\multicolumn{1}{|c|}{FR 7.5}&
		{Įvertinus kavinę, rezultatas yra siunčiamas į duomenų bazę ir kavinės reitingas yra atnaujinamas. Išeinama iš kavinės vertinimo lango.}&
		\multicolumn{1}{|c|}{BŪTINA}\\				
	\hline	
	
	\end{tabular}		
	\caption{Kavinės įvertinimo funkciniai reikalavimai}
	\label{table:KavinėsĮvertinimas}
	\end{table}


\end{center}
\pagebreak

\subsection{Kavinės planavimas}
Šiame poskyryje pateikti reikalavimai, susiję su kavinės planavimu (žr.~\ref{table:KavinėsPlanavimas} lentelę ).
\begin{center}
	\begin{table}[H]
	\begin{tabular}{|p{2cm}|p{12,5cm}|p{3,5cm}|}
	
	\hline
	    \rowcolor{lightgray}
		\multicolumn{3}{|c|}{Kavinės planavimas}\\
		
	\hline
		\multicolumn{1}{|c|}{{\bfseries Kodas}}&
		\multicolumn{1}{|c|}{{\bfseries Reikalavimas}}&
		\multicolumn{1}{|c|}{{\bfseries Svarba}}\\

	\hline
		\multicolumn{1}{|c|}{FR 8.1}&
		{Kavinės valdytojas gali vilkti toliau nurodytas figūras ant pateikto standartinio kavinės plano:
		\begin{itemize}
			\item Įvairūs stalai.
			\item Baro/Kasos stalas.
			\item Kėdės.
			\item Suolai/Sofos.
			\item Iėjimas.
			\item Langai
			\item Dekoratyviniai objektai.
		\end{itemize}}&
		\multicolumn{1}{|c|}{BŪTINA}\\

	\hline
	
		\multicolumn{1}{|c|}{FR 8.2}&
		{Nuvilkus stalą - paprašoma įvesti kiek žmonių gali susėsti prie to stalo.Po to nurodoma pridėti pakankamai kėdžių arba suolų sofų, kol patenkinamas vietų skaičius.}&
		\multicolumn{1}{|c|}{BŪTINA}\\				
	\hline
	
		\multicolumn{1}{|c|}{FR 8.3}&
		{Įvedus kiek žmonių gali sedėti prie stalo nurodoma kiek reikia pridėti kėdžių arba suolų sofų, kol patenkinamas vietų skaičius.}&
		\multicolumn{1}{|c|}{BŪTINA}\\				
	\hline
	
		\multicolumn{1}{|c|}{FR 8.4}&
		{Privaloma pridėti įėjimą.}&
		\multicolumn{1}{|c|}{BŪTINA}\\				
	\hline
	
		\multicolumn{1}{|c|}{FR 8.5}&
		{Privaloma pridėti barą arba kasos stalą.}&
		\multicolumn{1}{|c|}{BŪTINA}\\				
	\hline
	
		\multicolumn{1}{|c|}{FR 8.6}&
		{Valdytojas gali išsaugoti kavinės planą.}&
		\multicolumn{1}{|c|}{BŪTINA}\\				
	\hline
	
		\multicolumn{1}{|c|}{FR 8.7}&
		{Valdytojo išsaugotas kavinės planas patalpinamas į duomenų bazę.}&
		\multicolumn{1}{|c|}{BŪTINA}\\				
	\hline
	
	\end{tabular}
	\caption{Kavinės planavimo funkciniai reikalavimai}
	\label{table:KavinėsPlanavimas}
	\end{table}
	
	
\end{center}

\pagebreak

\subsection{Kavinės pridėjimas}
\begin{center}
	\begin{table}[H]
	\begin{tabular}{|p{2cm}|p{12,5cm}|p{3,5cm}|}
	
	\hline
	    \rowcolor{lightgray}
		\multicolumn{3}{|c|}{Kavinės pridėjimas}\\
		
	\hline
		\multicolumn{1}{|c|}{{\bfseries Kodas}}&
		\multicolumn{1}{|c|}{{\bfseries Reikalavimas}}&
		\multicolumn{1}{|c|}{{\bfseries Svarba}}\\

	\hline
		\multicolumn{1}{|c|}{FR 9.1}&
		{Jeigu vartotojas dar nėra sukūręs nei vienos kavinės, pridėdamas naują kavinę, jis automatiškai tampa valdytojų.}&
		\multicolumn{1}{|c|}{BŪTINA}\\	

	\hline
		\multicolumn{1}{|c|}{FR 9.2}&
		{Vartotojas arba valdytojas gali pridėti naują kavinę, įvesdamas tam tikrus duomenis:
			\begin{itemize}
				\item Privaloma įvesti kavinės pavadinimą.
				\item Privaloma nurodyti kavinės adresą.
				\item Būtina įvesti staliukų skaičių. (bent 1 staliukas).
				\item Privaloma nurodyti kavinės darbo laiką darbo dienomis ir savaitgaliais.
				\item Galima papildomai pridėti savininko telefono numerį.
			\end{itemize}}&
		\multicolumn{1}{|c|}{BŪTINA}\\	

	\hline
		\multicolumn{1}{|c|}{FR 9.3}&
		{Kavinės įvedamų duomenų formatai
			\begin{itemize}
				\item Kavinės pavadinimas (String).
				\item Kavinės adresas (String).
				\item Staliukų skaičius (Int, $>$0).
				\item Darbo dienomis (String).
				\item Šeštadieniais (String).
				\item Sekmadieniais (String).
			\end{itemize}}&
		\multicolumn{1}{|c|}{BŪTINA}\\	

	\hline
		\multicolumn{1}{|c|}{FR 9.4}&
		{Sėkmingai pridėjus kavinę pranešama informacija apie sėkmingai pridėtą kavinę.}&
		\multicolumn{1}{|c|}{BŪTINA}\\	

	\hline
		\multicolumn{1}{|c|}{FR 9.5}&
		{Sėkmingai pridėjus kavinę, jos duomenys išsaugomi duomenų bazėje.}&
		\multicolumn{1}{|c|}{BŪTINA}\\	

	\hline
		\multicolumn{1}{|c|}{FR 9.6}&
		{Valdytojas gali atšaukti kavinės pridėjimą.}&
		\multicolumn{1}{|c|}{BŪTINA}\\	

	\hline
		\multicolumn{1}{|c|}{FR 9.7}&
		{Sėkmingai atšaukus kavinės pridėjimą, automatiškai išeinama iš kavinės pridėjimo lango.}&
		\multicolumn{1}{|c|}{BŪTINA}\\	

	\hline
	
	
	
	\end{tabular}
	
	\end{table}

\end{center}


\subsection{Kavinės informacijos keitimas}
\begin{center}
	\begin{table}[H]
	\begin{tabular}{|p{2cm}|p{12,5cm}|p{3,5cm}|}
	\hline
	    \rowcolor{lightgray}
		\multicolumn{3}{|c|}{Kavinės informacijos keitimas}\\
		
	\hline
		\multicolumn{1}{|c|}{{\bfseries Kodas}}&
		\multicolumn{1}{|c|}{{\bfseries Reikalavimas}}&
		\multicolumn{1}{|c|}{{\bfseries Svarba}}\\
	\hline 	
		\multicolumn{1}{|c|}{FR 10.1}&
		{Tik valdytojas turi galimybę pakeisti savo pridėtų kavinių informaciją.}&
		\multicolumn{1}{|c|}{BŪTINA}\\
	
	\hline 	
		\multicolumn{1}{|c|}{FR 10.2}&
		{Kavinės duomenys, kuriuos gali valdytojas pakeisti:
		\begin{itemize}
			\item Kavinės pavadinimas.
			\item Kavinės adresas.
			\item Staliukų skaičius.
			\item Telefono numeris
		\end{itemize}}&
		\multicolumn{1}{|c|}{BŪTINA}\\
	
	\hline 	
		\multicolumn{1}{|c|}{FR 10.3}&
		{Sėkmingai pakeitus kavinės informacija, pranešama apie sėkmingai išsaugotus kavinės duomenis.}&
		\multicolumn{1}{|c|}{BŪTINA}\\
	
	\hline 	
		\multicolumn{1}{|c|}{FR 10.4}&
		{Sėkmingai pakeitus kavinės informacija, seni kavinės duomenys duomenų bazėje pakeičiami naujais.}&
		\multicolumn{1}{|c|}{BŪTINA}\\
	
	\hline  
	
	
	\end{tabular}
	\end{table}

\end{center}

\subsection{Premium kliento galimybės}

\begin{center}
	\begin{table}[H]
	\begin{tabular}{|p{2cm}|p{12,5cm}|p{3,5cm}|}
	\hline
	    \rowcolor{lightgray}
		\multicolumn{3}{|c|}{Premium kliento galimybės}\\
		
	\hline
		\multicolumn{1}{|c|}{{\bfseries Kodas}}&
		\multicolumn{1}{|c|}{{\bfseries Reikalavimas}}&
		\multicolumn{1}{|c|}{{\bfseries Svarba}}\\
	\hline 	
		\multicolumn{1}{|c|}{FR 11.1}&
		{Kiekvienas vartotojas turi galimybę užsisakyti Premium klientą.}&
		\multicolumn{1}{|c|}{BŪTINA}\\
				
	\hline 	
		\multicolumn{1}{|c|}{FR 11.2}&
		{Premium vartotojas turi greitas staliuko rezervavimą}&
		\multicolumn{1}{|c|}{BŪTINA}\\
				
	\hline 	
		\multicolumn{1}{|c|}{FR 11.3}&
		{Premium vartotojui parodomos arčiausiai esamos kavinės su laisvų staliukų skaičiais ir galima pasirinkti, kurioje kavinėje rezervuoti staliuką.}&
		\multicolumn{1}{|c|}{BŪTINA}\\
				
	\hline 	
		\multicolumn{1}{|c|}{FR 11.4}&
		{Premium vartotojui parodomos arčiausiai esamos kavinės su laisvų staliukų skaičiais ir galima pasirinkti, kurioje kavinėje rezervuoti staliuką.Jeigu šiuo metu nėra laisvų staliukų, apie tai pranešama.}&
		\multicolumn{1}{|c|}{BŪTINA}\\
				
	\hline 	
		\multicolumn{1}{|c|}{FR 11.5}&
		{Tik „Premium“ vartotojas gali rezervuoti paskutinį kavinės staliuką.}&
		\multicolumn{1}{|c|}{BŪTINA}\\
				
	\hline

	\end{tabular}
	
	\end{table}

\end{center}

\pagebreak
%NEFUNKCINIAI REIKALAVIMAI
\section{Nefunkciniai reikalavimai}

\subsection{Vidiniu interfeisų reikalavimai}

\subsubsection{OS naudojimo reikalavimai}

\begin{center}
	\begin{table}[H]
	\begin{tabular}{|p{2cm}|p{12,5cm}|p{3,5cm}|}
	\hline
	    \rowcolor{lightgray}
		\multicolumn{3}{|c|}{Apsaugos reikalavimai}\\
		
	\hline
		\multicolumn{1}{|c|}{{\bfseries Kodas}}&
		\multicolumn{1}{|c|}{{\bfseries Reikalavimas}}&
		\multicolumn{1}{|c|}{{\bfseries Svarba}}\\
	\hline 	
		\multicolumn{1}{|c|}{NFR 1.1.1}&
		{Išmanusis prietaisas turi įdiegtą Windows OS.}&
		\multicolumn{1}{|c|}{BŪTINA}\\	
	
	\hline 	
		\multicolumn{1}{|c|}{NFR 1.1.2}&
		{Išmanusis prietaisas privalo turėti interneto prieigą darbui su duomenų baze.}&
		\multicolumn{1}{|c|}{BŪTINA}\\	
	
	\hline 	
		\multicolumn{1}{|c|}{NFR 1.1.3}&
		{Išmanusis prietaisas privalo turėti GPS modulį.}&
		\multicolumn{1}{|c|}{BŪTINA}\\	
	
	\hline 	 	
	\end{tabular}
	\end{table}

\end{center}

\subsubsection{Sąveikos su DB reikalavimai}

\begin{center}
	\begin{table}[H]
	\begin{tabular}{|p{2cm}|p{12,5cm}|p{3,5cm}|}
	\hline
	    \rowcolor{lightgray}
		\multicolumn{3}{|c|}{Sąveikos su DB reikalavimai}\\
		
	\hline
		\multicolumn{1}{|c|}{{\bfseries Kodas}}&
		\multicolumn{1}{|c|}{{\bfseries Reikalavimas}}&
		\multicolumn{1}{|c|}{{\bfseries Svarba}}\\
	\hline 	
		\multicolumn{1}{|c|}{NFR 1.2.1}&
		{Duomenų bazėje turi būti saugomi vartotojų prisijungimo duomenys, informacija apie kavinės (pavadinimas, adresas, vietų skaičius, laisvų vietų skaičius, reitingas , darbo valandos).}&
		\multicolumn{1}{|c|}{BŪTINA}\\	
	
	\hline 	
		\multicolumn{1}{|c|}{NFR 1.2.2}&
		{Duomenys saugomi realiciniu būdu, naudojant MySQL duomenų bazių valdymo sistemą.}&
		\multicolumn{1}{|c|}{BŪTINA}\\	
	
	\hline  	 	
	\end{tabular}
	\end{table}

\end{center}

\subsubsection{Dokumentų mainų reikalavimai}

\begin{center}
	\begin{table}[H]
	\begin{tabular}{|p{2cm}|p{12,5cm}|p{3,5cm}|}
	\hline
	    \rowcolor{lightgray}
		\multicolumn{3}{|c|}{Dokumentų mainų reikalavimai}\\
		
	\hline
		\multicolumn{1}{|c|}{{\bfseries Kodas}}&
		\multicolumn{1}{|c|}{{\bfseries Reikalavimas}}&
		\multicolumn{1}{|c|}{{\bfseries Svarba}}\\
	\hline 	
		\multicolumn{1}{|c|}{NFR 1.3.1}&
		{Aplikacija su serveriu siunčia bei pateikia duomenis vienu iš šių formatų:
			\begin{itemize}
			\item JSON.
			\item String.
			\end{itemize}}&
		\multicolumn{1}{|c|}{BŪTINA}\\	
	
	\hline 	 	 	
	\end{tabular}
	\end{table}

\end{center}

\pagebreak

\subsubsection{Sąveikos su kitomis programomis reikalavimai}

\begin{center}
	\begin{table}[H]
	\begin{tabular}{|p{2cm}|p{12,5cm}|p{3,5cm}|}
	\hline
	    \rowcolor{lightgray}
		\multicolumn{3}{|c|}{Sąveikos su kitomis programomis reikalavimai}\\
		
	\hline
		\multicolumn{1}{|c|}{{\bfseries Kodas}}&
		\multicolumn{1}{|c|}{{\bfseries Reikalavimas}}&
		\multicolumn{1}{|c|}{{\bfseries Svarba}}\\
	\hline 	
		\multicolumn{1}{|c|}{NFR 1.4.1}&
		{Programa naudosis prietaiso navigacine GPS Sistema.}&
		\multicolumn{1}{|c|}{BŪTINA}\\	
	
	\hline 	
		\multicolumn{1}{|c|}{NFR 1.4.2}&
		{Programoje yra integruoti Google Maps žemėlapiai}&
		\multicolumn{1}{|c|}{BŪTINA}\\	
	
	\hline 	 	 	
	\end{tabular}
	\end{table}

\end{center}

\subsubsection{Programavimo aplinkos reikalavimai}

\begin{center}
	\begin{table}[H]
	\begin{tabular}{|p{2cm}|p{12,5cm}|p{3,5cm}|}
	\hline
	    \rowcolor{lightgray}
		\multicolumn{3}{|c|}{Programavimo aplinkos reikalavimai}\\
		
	\hline
		\multicolumn{1}{|c|}{{\bfseries Kodas}}&
		\multicolumn{1}{|c|}{{\bfseries Reikalavimas}}&
		\multicolumn{1}{|c|}{{\bfseries Svarba}}\\
	\hline 	
		\multicolumn{1}{|c|}{NFR 1.5.1}&
		{Programa kuriama naudojant C\# programavimo kalbą.}&
		\multicolumn{1}{|c|}{BŪTINA}\\	
	
	\hline 	
		\multicolumn{1}{|c|}{NFR 1.5.2}&
		{Naudojama Github repozitorija }&
		\multicolumn{1}{|p{1.5cm}|}{PAGEIDAUTINA}\\		
	
	\hline 	
		\multicolumn{1}{|c|}{NFR 1.5.3}&
		{Naudojama Visual Studio programavimo aplinka.}&
		\multicolumn{1}{|p{1.5cm}|}{PAGEIDAUTINA}\\		
	
	\hline 	 	 	
	\end{tabular}
	\end{table}

\end{center}

\subsection{Veikimo reikalavimai}

\subsubsection{Vaizdavimo reikalavimai}

\begin{center}
	\begin{table}[H]
	\begin{tabular}{|p{2cm}|p{12,5cm}|p{3,5cm}|}
	\hline
	    \rowcolor{lightgray}
		\multicolumn{3}{|c|}{Vaizdavimo reikalavimai}\\
		
	\hline
		\multicolumn{1}{|c|}{{\bfseries Kodas}}&
		\multicolumn{1}{|c|}{{\bfseries Reikalavimas}}&
		\multicolumn{1}{|c|}{{\bfseries Svarba}}\\
	\hline 	
		\multicolumn{1}{|c|}{NFR 2.1.1}&
		{Programa yra vaizduojama naudojant standartinį .NET vaizdavimo šabloną Windows Forms.}&
		\multicolumn{1}{|c|}{BŪTINA}\\	
	
	\hline 	
		\multicolumn{1}{|c|}{NFR 2.1.2}&
		{Data yra vaizduojama YYYY-MM-DD hh:mm formatu, kur YY – metai, MM – mėnuo, DD – diena, hh – valandos, mm - minutės.}&
		\multicolumn{1}{|c|}{BŪTINA}\\	
	
	\hline 	
		\multicolumn{1}{|c|}{NFR 2.1.3}&
		{Kavinės pavadinimas – ne ilgiau nei 30 simbolių.}&
		\multicolumn{1}{|c|}{BŪTINA}\\	
	
	\hline  	
		\multicolumn{1}{|c|}{NFR 2.1.4}&
		{Visos sistemoje registruotos kavinės yra pateiktos lentelėje.}&
		\multicolumn{1}{|p{1.5cm}|}{PAGEIDAUTINA}\\		
	
	\hline 	 	 	
	\end{tabular}
	\end{table}

\end{center}

\subsubsection{Skaičiavimo tikslumo reikalavimai}

\begin{center}
	\begin{table}[H]
	\begin{tabular}{|p{2cm}|p{12,5cm}|p{3,5cm}|}
	\hline
	    \rowcolor{lightgray}
		\multicolumn{3}{|c|}{Skaičiavimo tikslumo reikalavimai}\\
		
	\hline
		\multicolumn{1}{|c|}{{\bfseries Kodas}}&
		\multicolumn{1}{|c|}{{\bfseries Reikalavimas}}&
		\multicolumn{1}{|c|}{{\bfseries Svarba}}\\
	\hline 	
		\multicolumn{1}{|c|}{NFR 2.2.1}&
		{Skaičiuojant vartotojo esamas koordinatės paklaida negali būti didesnė nei 30m.}&
		\multicolumn{1}{|c|}{BŪTINA}\\	
	\hline 	 	 	
	\end{tabular}
	\end{table}

\end{center}

\subsubsection{Patikimumo reikalavimai}

\begin{center}
	\begin{table}[H]
	\begin{tabular}{|p{2cm}|p{12,5cm}|p{3,5cm}|}
	\hline
	    \rowcolor{lightgray}
		\multicolumn{3}{|c|}{Patikimumo reikalavimai}\\
		
	\hline
		\multicolumn{1}{|c|}{{\bfseries Kodas}}&
		\multicolumn{1}{|c|}{{\bfseries Reikalavimas}}&
		\multicolumn{1}{|c|}{{\bfseries Svarba}}\\
	\hline 	
		\multicolumn{1}{|c|}{NFR 2.3.1}&
		{Sistema turi veikti 95\% planuoto veikimo laiko. }&
		\multicolumn{1}{|c|}{BŪTINA}\\	
	\hline 	 	 	
	\end{tabular}
	\end{table}

\end{center}



\subsubsection{Robatiškimo reikalavimai}

\begin{center}
	\begin{table}[H]
	\begin{tabular}{|p{2cm}|p{12,5cm}|p{3,5cm}|}
	\hline
	    \rowcolor{lightgray}
		\multicolumn{3}{|c|}{Robatiškumo reikalavimai}\\
		
	\hline
		\multicolumn{1}{|c|}{{\bfseries Kodas}}&
		\multicolumn{1}{|c|}{{\bfseries Reikalavimas}}&
		\multicolumn{1}{|c|}{{\bfseries Svarba}}\\
	\hline 	
		\multicolumn{1}{|c|}{NFR 2.4.1}&
		{Keičiant ar pildant programėlės įvestis. Yra nedelsiant modifikuojama programėlės duomenų bazė.}&
		\multicolumn{1}{|c|}{BŪTINA}\\
		
	\hline 	
		\multicolumn{1}{|c|}{NFR 2.4.2}&
		{Įvykus sistemos klaidai susijusiai su užsakymais, užsakymas yra nutraukiamas ir apie tai pranešama užsakovui.}&
		\multicolumn{1}{|c|}{BŪTINA}\\
		
	\hline 	
		\multicolumn{1}{|c|}{NFR 2.4.3}&
		{Bandant pildyti laukus ne pagal pateiktus reikalavimus, užklausa negali būti įvykdyta.}&
		\multicolumn{1}{|c|}{BŪTINA}\\
		
	\hline
	
	
	\end{tabular}
	\end{table}

\end{center}


\subsubsection{Laikas, reikalingas atstatyti programos veikimą}
\begin{center}
	\begin{table}[H]
	\begin{tabular}{|p{2cm}|p{12,5cm}|p{3,5cm}|}
	\hline
	    \rowcolor{lightgray}
		\multicolumn{3}{|c|}{Laikas, reikalingas atstatyti programos veikimą}\\
		
	\hline
		\multicolumn{1}{|c|}{{\bfseries Kodas}}&
		\multicolumn{1}{|c|}{{\bfseries Reikalavimas}}&
		\multicolumn{1}{|c|}{{\bfseries Svarba}}\\
	\hline 	
		\multicolumn{1}{|c|}{NFR 2.5.1}&
		{Atsiradę programėlės trikdžiai turi būti sutvarkyti per 3 darbo dienas, komandos atsakingos už šios programėlės veikimą.}&
		\multicolumn{1}{|c|}{BŪTINA}\\		
	\hline
	
	
	\end{tabular}
	\end{table}

\end{center}


\subsubsection{Našumo reikalavimai}
\begin{center}
	\begin{table}[H]
	\begin{tabular}{|p{2cm}|p{12,5cm}|p{3,5cm}|}
	\hline
	    \rowcolor{lightgray}
		\multicolumn{3}{|c|}{Našumo reikalavimai}\\
		
	\hline
		\multicolumn{1}{|c|}{{\bfseries Kodas}}&
		\multicolumn{1}{|c|}{{\bfseries Reikalavimas}}&
		\multicolumn{1}{|c|}{{\bfseries Svarba}}\\
	\hline 	
		\multicolumn{1}{|c|}{NFR 2.6.1}&
		{Programa neturi naudoti daugiau nei 50 proc. procesoriaus pajėgumo.}&
		\multicolumn{1}{|p{1.5cm}|}{PAGEIDAUTINA}\\	
	\hline 	
		\multicolumn{1}{|c|}{NFR 2.6.2}&
		{Konkrečiai kavinės staliuko paieškai ir rezervacijai duomenų Bazėje turi būti sugaišta ne ilgiau nei 7 sekundės.}&
		\multicolumn{1}{|p{1.5cm}|}{PAGEIDAUTINA}\\	
	\hline 	
		\multicolumn{1}{|c|}{NFR 2.6.3}&
		{Kiekviena užklausa turi būti aptarnauta per ne ilgiau kaip 10 sekundžių.}&
		\multicolumn{1}{|p{1.5cm}|}{PAGEIDAUTINA}\\		
	\hline
	
	
	\end{tabular}
	\end{table}

\end{center}

\pagebreak

\subsection{Diegimo reikalavimai}

\subsubsection{Ruošinio reikalavimai}
\begin{center}
	\begin{table}[H]
	\begin{tabular}{|p{2cm}|p{12,5cm}|p{3,5cm}|}
	\hline
	    \rowcolor{lightgray}
		\multicolumn{3}{|c|}{Ruošinio reikalavimai}\\
		
	\hline
		\multicolumn{1}{|c|}{{\bfseries Kodas}}&
		\multicolumn{1}{|c|}{{\bfseries Reikalavimas}}&
		\multicolumn{1}{|c|}{{\bfseries Svarba}}\\
	\hline 	
		\multicolumn{1}{|c|}{NFR 3.1.1}&
		{Programėlė pilnai veikia prisijungiant prie interneto iš bet kurio IP adreso.}&
		\multicolumn{1}{|p{1.5cm}|}{PAGEIDAUTINA}\\	
	
	\hline
	
	
	\end{tabular}
	\end{table}

\end{center}

\subsubsection{Instaliavimo reikalavimai }
\begin{center}
	\begin{table}[H]
	\begin{tabular}{|p{2cm}|p{12,5cm}|p{3,5cm}|}
	\hline
	    \rowcolor{lightgray}
		\multicolumn{3}{|c|}{Instaliavimo reikalavimai}\\
		
	\hline
		\multicolumn{1}{|c|}{{\bfseries Kodas}}&
		\multicolumn{1}{|c|}{{\bfseries Reikalavimas}}&
		\multicolumn{1}{|c|}{{\bfseries Svarba}}\\
	\hline 	
		\multicolumn{1}{|c|}{NFR 3.2.1}&
		{Norėdamas įdiegti aplikaciją vartotojas privalo duoti sutikimą dėl duomenų gavimo internetu.}&
		\multicolumn{1}{|c|}{BŪTINA}\\	
	
	\hline 	
		\multicolumn{1}{|c|}{NFR 3.2.2}&
		{Sklandžiam aplikacijos diegimui ir veikimui reikia 1MB ROM. 2GB RAM.}&
		\multicolumn{1}{|p{1.5cm}|}{PAGEIDAUTINA}\\	
	
	\hline
	
	
	\end{tabular}
	\end{table}

\end{center}

\subsubsection{Pradinio DB kaupimo reikalavimai}
\begin{center}
	\begin{table}[H]
	\begin{tabular}{|p{2cm}|p{12,5cm}|p{3,5cm}|}
	\hline
	    \rowcolor{lightgray}
		\multicolumn{3}{|c|}{Pradinio DB kaupimo reikalavimai}\\
		
	\hline
		\multicolumn{1}{|c|}{{\bfseries Kodas}}&
		\multicolumn{1}{|c|}{{\bfseries Reikalavimas}}&
		\multicolumn{1}{|c|}{{\bfseries Svarba}}\\
	\hline 	
		\multicolumn{1}{|c|}{NFR 3.3.1}&
		{Sistemos duomenų bazė turi turėti:
			\begin{itemize}
				\item Kavinių lentelė turi bent 10 pradinių užpildytų eilučių su informacija apie Kavines. Juos įveda įgaliotas įmonės darbuotojas naudodamasis administratoriaus interfeisu.
				\item Kavinių staliukų lentelė turi saugoti visų esančių staliukų sąrašą.
				\item Vartotojų lentelė turi saugoti klientų sąrašą.
				\item Administratorių lentelė turi turėti bent po vieną administratorių kiekvienai kavinei.
			\end{itemize}}&
		\multicolumn{1}{|c|}{BŪTINA}\\	
	
	\hline 	
	
	\end{tabular}
	\end{table}

\end{center}

\subsubsection{Sistemos įsisavinamumo reikalavimai}
\begin{center}
	\begin{table}[H]
	\begin{tabular}{|p{2cm}|p{12,5cm}|p{3,5cm}|}
	\hline
	    \rowcolor{lightgray}
		\multicolumn{3}{|c|}{Sistemos įsisavinamumo reikalavimai}\\
		
	\hline
		\multicolumn{1}{|c|}{{\bfseries Kodas}}&
		\multicolumn{1}{|c|}{{\bfseries Reikalavimas}}&
		\multicolumn{1}{|c|}{{\bfseries Svarba}}\\
	\hline 	
		\multicolumn{1}{|c|}{NFR 3.4.1}&
		{Negali būti klaidinančių nuorodų.}&
		\multicolumn{1}{|c|}{BŪTINA}\\	
	
	\hline 	
		\multicolumn{1}{|c|}{NFR 3.4.2}&
		{Pirmą kartą paleidus programą reikia susikurti vartotojo paskirą, tam reikia įvesti:
			\begin{itemize}
				\item Slapyvardį.
				\item Slaptažodį.
				\item Telefono numerį.
				\item Pašto adresą.
				\item Nurodyti ar naudotojas yra klientas, ar kavinės savininkas.
			\end{itemize}}&
		\multicolumn{1}{|c|}{BŪTINA}\\	
	
	\hline 	
	
	\end{tabular}
	\end{table}

\end{center}

\subsection{Ekonominiai ribojimai}
\begin{center}
	\begin{table}[H]
	\begin{tabular}{|p{2cm}|p{12,5cm}|p{3,5cm}|}
	\hline
	    \rowcolor{lightgray}
		\multicolumn{3}{|c|}{Ekonominiai ribojimai}\\
		
	\hline
		\multicolumn{1}{|c|}{{\bfseries Kodas}}&
		\multicolumn{1}{|c|}{{\bfseries Reikalavimas}}&
		\multicolumn{1}{|c|}{{\bfseries Svarba}}\\
	\hline 	
		\multicolumn{1}{|c|}{NFR 4.1}&
		{Ši programėlė yra nemokama.}&
		\multicolumn{1}{|p{1.5cm}|}{PAGEIDAUTINA}\\	
	
	\hline
	
	
	\end{tabular}
	\end{table}

\end{center}

\pagebreak

\subsection{Aptarnavimo ir priežiūros reikalavimai}
\begin{center}
	\begin{table}[H]
	\begin{tabular}{|p{2cm}|p{12,5cm}|p{3,5cm}|}
	\hline
	    \rowcolor{lightgray}
		\multicolumn{3}{|c|}{Aptarnavimo ir priežiūros reikalavimai}\\
		
	\hline
		\multicolumn{1}{|c|}{{\bfseries Kodas}}&
		\multicolumn{1}{|c|}{{\bfseries Reikalavimas}}&
		\multicolumn{1}{|c|}{{\bfseries Svarba}}\\
	\hline 	
		\multicolumn{1}{|c|}{NFR 5.1}&
		{Aplikacijoje atsiradusi klaida turi būti ištaisyta per 3 darbo dienas.}&
		\multicolumn{1}{|c|}{BŪTINA}\\	
	
	\hline 	
		\multicolumn{1}{|c|}{NFR 5.2}&
		{Visi vartotojo atliekami veiksmai programėlėje turi būti sekami ir saugomi laikinojoje duomenų bazėje tam, kad jei vartotojas esamos prisijungimo sesijos metu atrastų tinklapio spragą, visi jo atlikti veiksmai, kurie galėjo tai sukelti, būtų išsiųsti kaip klaidos pranešimas ir darbuotojai, atsakingi už tinklapio sklandų veikimą galėtų išanalizuoti spragą ir ją panaikinti.}&
		\multicolumn{1}{|p{1.5cm}|}{PAGEIDAUTINA}\\
	
	\hline 	
		\multicolumn{1}{|c|}{NFR 5.3}&
		{Prieš praplečiant vartotojų galimybes būtina atlikti automatinius testus, kurie susimuliuoja vartotoją bandantį atlikti visas naujas jam pasiekiamas funkcijas tam, kad būtų išlaikytas programėlės funkcionalumas. Susidūrus su klaidomis, plėtimas turi būti atidėtas iki kol bus ištaisytos klaidos.}&
		\multicolumn{1}{|c|}{BŪTINA}\\	
		
	\hline 	
	
	\end{tabular}
	\end{table}

\end{center}

\subsection{Tiražuojamumo reikalavimai}
\begin{center}
	\begin{table}[H]
	\begin{tabular}{|p{2cm}|p{12,5cm}|p{3,5cm}|}
	\hline
	    \rowcolor{lightgray}
		\multicolumn{3}{|c|}{Tiražuojamumo reikalavimai}\\
		
	\hline
		\multicolumn{1}{|c|}{{\bfseries Kodas}}&
		\multicolumn{1}{|c|}{{\bfseries Reikalavimas}}&
		\multicolumn{1}{|c|}{{\bfseries Svarba}}\\
	\hline 	
		\multicolumn{1}{|c|}{NFR 6.1}&
		{Programa turi veikti Windows operacinėje sistemoje.}&
		\multicolumn{1}{|c|}{BŪTINA}\\	
	
	\hline 	
	\end{tabular}
	\end{table}

\end{center}

\subsection{Apsaugos reikalavimai}
\begin{center}
	\begin{table}[H]
	\begin{tabular}{|p{2cm}|p{12,5cm}|p{3,5cm}|}
	\hline
	    \rowcolor{lightgray}
		\multicolumn{3}{|c|}{Apsaugos reikalavimai}\\
		
	\hline
		\multicolumn{1}{|c|}{{\bfseries Kodas}}&
		\multicolumn{1}{|c|}{{\bfseries Reikalavimas}}&
		\multicolumn{1}{|c|}{{\bfseries Svarba}}\\
	\hline 	
		\multicolumn{1}{|c|}{NFR 7.1}&
		{Yra saugomi ir šifruojami kliento prisijungimo duomenys.}&
		\multicolumn{1}{|c|}{BŪTINA}\\	
	
	\hline 	
	\end{tabular}
	\end{table}

\end{center}

\subsection{Juridiniai reikalavimai}
\begin{center}
	\begin{table}[H]
	\begin{tabular}{|p{2cm}|p{12,5cm}|p{3,5cm}|}
	\hline
	    \rowcolor{lightgray}
		\multicolumn{3}{|c|}{Juridiniai reikalavimai}\\
		
	\hline
		\multicolumn{1}{|c|}{{\bfseries Kodas}}&
		\multicolumn{1}{|c|}{{\bfseries Reikalavimas}}&
		\multicolumn{1}{|c|}{{\bfseries Svarba}}\\
	\hline 	
		\multicolumn{1}{|c|}{NFR 8.1}&
		{Aplikacija turi nepažeisti Lietuvos respublikos asmens duomenų teisinės apsaugos įstatymo.}&
		\multicolumn{1}{|c|}{BŪTINA}\\	
	
	\hline 	
		\multicolumn{1}{|c|}{NFR 8.2}&
		{Vartotojas registracijos metu turi susipažinti ir su tikti su aplikacijos naudojimo sąlygomis.}&
		\multicolumn{1}{|c|}{BŪTINA}\\	
	
	\hline 	
	\end{tabular}
	\end{table}

\end{center}

\pagebreak

%?SITO REIKIA?
\section{Literatūros sąrašas}

%?SITO REIKIA?
\section{Priedai}
\end{document}
