\documentclass{VUMIFPSkursinis}
\usepackage{algorithmicx}
\usepackage{algorithm}
\usepackage{algpseudocode}
\usepackage{amsfonts}
\usepackage{amsmath}
\usepackage{bm}
\usepackage{caption}
\usepackage{color}
\usepackage{float}
\usepackage{graphicx}
\usepackage{listings}
\usepackage{subfig}
\usepackage{wrapfig}
\usepackage{pdflscape} %Keep it to pdflscape or I can't rotate my diagram (K.S.)
\usepackage{longtable}
\usepackage[table]{xcolor}
\usepackage{multirow}
\usepackage[usestackEOL]{stackengine}

\usepackage{enumitem}
%PAKEISTA, tarpai tarp sąrašo elementų
\setitemize{noitemsep,topsep=0pt,parsep=0pt,partopsep=0pt}
\setenumerate{noitemsep,topsep=0pt,parsep=0pt,partopsep=0pt}

% Titulinio aprašas
\university{Vilniaus universitetas}
\faculty{Matematikos ir informatikos fakultetas}
\department{Programų sistemų katedra}
\papertype{Programų sistemų inžinerijos I laboratorinis darbas Nr. 2}
\title{Reikalavimų specifikacija}
\titleineng{}
\status{2 kurso 5 grupės studentai}
\author{Paulius Grigaliūnas}
\secondauthor{Karolis Staskevičius}
\thirdauthor{Modestas Dulevičius}
\fourthauthor{Albert Jurkoit}
\fifthauthor{Šarūnas Kazimieras Buteikis}
     

% \secondauthor{Vardonis Pavardonis}   % Pridėti antrą autorių
\supervisor{dr. Vytautas Valaitis}
\date{Vilnius – \the\year}

% Nustatymai
% \setmainfont{Palemonas}   % Pakeisti teksto šriftą į Palemonas (turi būti įdiegtas sistemoje)

\begin{document}
% PAKEISTA	
\maketitle
\cleardoublepage\pagenumbering{arabic}
\setcounter{page}{2}


%ANOTACIJA

\sectionnonum{ANOTACIJA}
{\bfseries Darbo tikslas:} Reikalavimų anotacija Plz kažkas parašykite :). 
\newline
\newline
\newline

%DARBO ATLIKO
\noindent
{\bfseries Darbą atliko:}
\newline
\newline
\newline
Paulius Grigaliūnas
\newline
paulius.grigaliunas.pg@gmail.com
\newline
\newline
\newline
Karolis Staskevičius
\newline
satelistas@gmail.com
\newline
\newline
\newline
Modestas Dulevičius
\newline
modux9@gmail.com
\newline
\newline
\newline
Albert Jurkoit
\newline
albert.jurkoit@mif.stud.vu.lt
\newline
\newline
\newline
Šarūnas Kazimieras Buteikis
\newline
sarunas.kazimieras.buteikis@gmail.com

%TURINYS(TOC)
\tableofcontents

%ĮVADAS
\sectionnonum{Įvadas}

Čionais bus mūsų įvadas. Vėliau jį užpildysime.
\newline

%FUNKCINIAI REIKALAVIMAI
\section{Funkciniai reikalavimai}
%PRADEDAME RASYTI LENTELES WOHOO

Šiame skyriuje bus nurodomi aplikacijos funkciniai reikalavimai ir jų įgyvendinimo svarba.

\subsection{Aplikacijos langai}

 Šiame poskyryje pateikti funkciniai reikalavimai, susiję su aplikacijos langais (žr.~\ref{table:AplikacijosLangai} lentelę).
 %APLIKACIJOS LANGAI
\begin{center}
	\begin{table}[H]
	\begin{tabular}{|p{2cm}|p{12,5cm}|p{3,5cm}|}
	\hline
	    \rowcolor{lightgray}
		\multicolumn{3}{|c|}{Aplikacijos langai}\\
		
	\hline
		\multicolumn{1}{|c|}{{\bfseries Kodas}}&
		\multicolumn{1}{|c|}{{\bfseries Reikalavimas}}&
		\multicolumn{1}{|c|}{{\bfseries Svarba}}\\

	\hline
		\multicolumn{1}{|c|}{FR 1.1} &
		Iš visų aplikacijos langų galima grįžti į prisijungimo langą &
		\multicolumn{1}{|c|}{BŪTINA}\\
	\hline
		\multicolumn{1}{|c|}{FR 1.2} &
		{Visi langai po prisijungimo turi turėti galimybę parodyti vartotojo \newline prisijungimo vardą/paštą.}&
		\multicolumn{1}{|c|}{BŪTINA}\\
	\hline
	
		\multicolumn{1}{|c|}{FR 1.3}&
		{Jeigu vartotojas turi rezervavęs staliuką - iš visų langų, išskyrus \newline prisijungimo, galima patekti į langą rodantį rezervacijos informaciją.}&
		\multicolumn{1}{|c|}{BŪTINA}\\
	\hline
		\multicolumn{1}{|c|}{FR 1.4}&
		{š prisijungimo lango galima patekti į registracijos ir meniu langą.}&
		\multicolumn{1}{|c|}{BŪTINA}\\
	\hline
		\multicolumn{1}{|c|}{FR 1.5}&
		{Iš meniu lango galima patekti į:\newline
		\begin{enumerate}
			\item Jei prisijungia kavinės administratorius:
				\begin{enumerate}
					\item “Keisti kavinės informaciją” langą.
					\item “Keisti kavinės išplanavimą” langą.
				\end{enumerate}
			\item Jei prisijungia kavinės darbuotojas:
				\begin{enumerate}
					\item “Lokali rezervacija” langą.
				\end{enumerate}
			\item Jei prisijungia klientas:
				\begin{enumerate}
					\item “Kavinės paieška” langas.
				\end{enumerate}
			\item Visi prisijungę vartotojai turi galimybę patekti į “Keisti vartotojo duomenis” langą.
			\item Visi langai turi turėti galimybę grįžti į ankstesnį langą.
		\end{enumerate}
		}&
		\multicolumn{1}{|c|}{BŪTINA}\\		
	\hline
	
	\end{tabular}
	\caption{Aplikacijos langų funkciniai reikalavimai.}
	\label{table:AplikacijosLangai}
	\end{table}
	
\end{center}

\pagebreak

\subsection{Vartotojo registracija}

Šiame poskyryje pateikti funkciniai reikalavimai, susiję su vartotojo registracija (žr.~\ref{table:VartotojoRegistracija} lentelę).
%Vartotojo prisijungimas
\begin{center}

	\begin{table}[H]
	\begin{tabular}{|p{2cm}|p{12,5cm}|p{3,5cm}|}
	\hline
	    \rowcolor{lightgray}
		\multicolumn{3}{|c|}{Vartotojo registracija}\\
		
	\hline
		\multicolumn{1}{|c|}{{\bfseries Kodas}}&
		\multicolumn{1}{|c|}{{\bfseries Reikalavimas}}&
		\multicolumn{1}{|c|}{{\bfseries Svarba}}\\

	\hline
	\multicolumn{1}{|c|}{FR 2.1}&
	{Vartotojas, norintis registruotis prie aplikacijos, turi nurodyti:
		\begin{itemize}
			\item Vardą.
			\item Pavardę.
			\item Slaptažodį.
			\item Telefono numerį.
			\item Elektroninį paštą.
		\end{itemize}}&		
	\multicolumn{1}{|c|}{BŪTINA}\\
	\hline
	
		\multicolumn{1}{|c|}{FR 2.2}&
		{Informuoti vartotoją apie sėkmingą registraciją}&
		\multicolumn{1}{|c|}{BŪTINA}\\	
	\hline
		\multicolumn{1}{|c|}{FR 2.3}&
		{Vartotojo duomenys išsaugomi duomenų bazėje}&
		\multicolumn{1}{|c|}{BŪTINA}\\	
	\hline	
		\multicolumn{1}{|c|}{FR 2.4}&
		{Tam, kad įvestas vartotojo el. pašto adresas būtų tinkamos formos, jis turi būti sudarytas iš abonento vardo, „@“ simbolio bei domeno adreso.}&
		\multicolumn{1}{|c|}{BŪTINA}\\	
	\hline
		\multicolumn{1}{|c|}{FR 2.5}&
		{Slaptažodis privalo būti sudarytas bent iš 8 simbolių, turi bent vieną skaitmenį.
}&
		\multicolumn{1}{|c|}{BŪTINA}\\		
	\hline
		\multicolumn{1}{|c|}{FR 2.6}&
		{Duomenų bazėje išsaugomi vartotojo duomenys turi būti string tipo }&
		\multicolumn{1}{|p{1.5cm}|}{PAGEIDAUTINA}\\		
	\hline
	
	\end{tabular}
	\caption{Vartotojo registracijos funkciniai reikalavimai}
	\label{table:VartotojoRegistracija}
	\end{table}


\end{center}

%NEFUNKCINIAI REIKALAVIMAI
\section{Nefunkciniai reikalavimai}



\begin{center}
\begin{tabular}{|p{2cm}|p{3cm}|p{7cm}|p{3cm}|}
\hline
\multicolumn{1}{|c|}{{\bfseries Kodas}}&
\multicolumn{2}{|c|}{{\bfseries Reikalavimas}}&
\multicolumn{1}{|c|}{{\bfseries Svarba}}\\
\hline
\rowcolor{lightgray} 
\multicolumn{4}{|c|}{\textit{Funkciniai reikalavimai, skirti aplikacijos vartotojams}}\\
\hline
\multirow{3}[30]{*}{FR 1} &
\multicolumn{2}{|l|}{BBB  AAA} &
bcd\\
\cline{2-4}
 & \multicolumn{2}{|l|}{\Centerstack[l]{\underline{Aplikacijos langai}\\ \\ 
 Čia apibrėžti pagrindiniai aplikacijoje esantys langai, į kuriuos \\
 gali patekti vartotojas. }} & \multirow{3}[3]{*}{SVARBU}\\\cline{2-3}%cline, nuo kurio column iki kurio brezti horizontalia linija
 & FR 1.1 & \Centerstack[l]{Pradinis aplikacijos langas - šiame lange \\vartotojas gali
 prisiregistruoti arba \\prisijungti prie aplikacijos.}&\\
\hline
\end{tabular}
\end{center}



%?SITO REIKIA?
\section{Literatūros sąrašas}

%?SITO REIKIA?
\section{Priedai}
\end{document}
