\documentclass{VUMIFPSkursinis}
\usepackage{algorithmicx}
\usepackage{algorithm}
\usepackage{algpseudocode}
\usepackage{amsfonts}
\usepackage{amsmath}
\usepackage{bm}
\usepackage{caption}
\usepackage{color}
\usepackage{float}
\usepackage{graphicx}
\usepackage{listings}
\usepackage{subfig}
\usepackage{wrapfig}
\usepackage{pdflscape} %Keep it to pdflscape or I can't rotate my diagram (K.S.)
\usepackage{longtable}
\usepackage[table]{xcolor}
\usepackage{multirow}
\usepackage[usestackEOL]{stackengine}

\usepackage{enumitem}
%PAKEISTA, tarpai tarp sąrašo elementų
\setitemize{noitemsep,topsep=0pt,parsep=0pt,partopsep=0pt}
\setenumerate{noitemsep,topsep=0pt,parsep=0pt,partopsep=0pt}

% Titulinio aprašas
\university{Vilniaus universitetas}
\faculty{Matematikos ir informatikos fakultetas}
\department{Programų sistemų katedra}
\papertype{Programų sistemų inžinerijos I laboratorinis darbas Nr. 2}
\title{Reikalavimų specifikacija}
\titleineng{}
\status{2 kurso 5 grupės studentai}
\author{Paulius Grigaliūnas}
\secondauthor{Karolis Staskevičius}
\thirdauthor{Modestas Dulevičius}
\fourthauthor{Albert Jurkoit}
\fifthauthor{Šarūnas Kazimieras Buteikis}
     

% \secondauthor{Vardonis Pavardonis}   % Pridėti antrą autorių
\supervisor{dr. Vytautas Valaitis}
\date{Vilnius – \the\year}

% Nustatymai
% \setmainfont{Palemonas}   % Pakeisti teksto šriftą į Palemonas (turi būti įdiegtas sistemoje)

\begin{document}
% PAKEISTA	
\maketitle
\cleardoublepage\pagenumbering{arabic}
\setcounter{page}{2}


%ANOTACIJA

\sectionnonum{ANOTACIJA}
{\bfseries Darbo tikslas:} Reikalavimų anotacija Plz kažkas parašykite :). 
\newline
\newline
\newline

%DARBO ATLIKO
\noindent
{\bfseries Darbą atliko:}
\newline
\newline
\newline
Paulius Grigaliūnas
\newline
paulius.grigaliunas.pg@gmail.com
\newline
\newline
\newline
Karolis Staskevičius
\newline
satelistas@gmail.com
\newline
\newline
\newline
Modestas Dulevičius
\newline
modux9@gmail.com
\newline
\newline
\newline
Albert Jurkoit
\newline
albert.jurkoit@mif.stud.vu.lt
\newline
\newline
\newline
Šarūnas Kazimieras Buteikis
\newline
sarunas.kazimieras.buteikis@gmail.com

%TURINYS(TOC)
\tableofcontents

%ĮVADAS
\sectionnonum{Įvadas}

Čionais bus mūsų įvadas. Vėliau jį užpildysime.
\newline

\section{Reikalavimų specifikacija}
%FUNKCINIAI REIKALAVIMAI
\subsection{Funkciniai reikalavimai}
%PRADEDAME RASYTI LENTELES WOHOO

Žemiau pateikta, kaip turėtų atrodyti mūsų lentelė :) Viskas bus ez, nėr ką čia veikt :Dąąąąąąąąąąąąąąąąąąąąą
\begin{center}
\begin{tabular}{|p{2cm}|p{3cm}|p{7cm}|p{3cm}|}
\hline
\multicolumn{1}{|c|}{{\bfseries Kodas}}&
\multicolumn{2}{|c|}{{\bfseries Reikalavimas}}&
\multicolumn{1}{|c|}{{\bfseries Svarba}}\\
\hline
\rowcolor{lightgray} 
\multicolumn{4}{|c|}{\textit{Funkciniai reikalavimai, skirti aplikacijos vartotojams}}\\
\hline
\multirow{3}{*}{FR1} &
\multicolumn{2}{|l|}{BBB  AAA} &
bcd\\
\cline{2-4}
 & \multicolumn{2}{|l|}{\Centerstack[l]{\underline{Aplikacijos langai}\\ \\ 
 Čia apibrėžti pagrindiniai aplikacijoje esantys langai, į kuriuos \\
 gali patekti vartotojas. }} & \multirow{3}{*}{SVARBU}\\\cline{2-3}%cline, nuo kurio column iki kurio brezti horizontalia linija
 & FR 1.1 & \Centerstack[l]{Pradinis aplikacijos langas - šiame lange \\vartotojas gali
 prisiregistruoti arba \\prisijungti prie aplikacijos.}&\\
\hline
\end{tabular}
\end{center}
%NEFUNKCINIAI REIKALAVIMAI
\subsection{Nefunkciniai reikalavimai}

%?SITO REIKIA?
\section{Literatūros sąrašas}

%?SITO REIKIA?
\section{Priedai}
\end{document}
