\documentclass{VUMIFPSkursinis}
\usepackage{algorithmicx}
\usepackage{algorithm}
\usepackage{algpseudocode}
\usepackage{amsfonts}
\usepackage{amsmath}
\usepackage{bm}
\usepackage{caption}
\usepackage{color}
\usepackage{float}
\usepackage{graphicx}
\usepackage{listings}
\usepackage{subfig}
\usepackage{wrapfig}
\usepackage{pdflscape} %Keep it to pdflscape or I can't rotate my diagram (K.S.)
\usepackage{longtable}
\usepackage[table]{xcolor}
\usepackage{multirow}
\usepackage[usestackEOL]{stackengine}
\usepackage{longtable}
\usepackage{subfig}
\usepackage{wrapfig}


\usepackage{enumitem}
%PAKEISTA, tarpai tarp sąrašo elementų
\setitemize{noitemsep,topsep=0pt,parsep=0pt,partopsep=0pt}
\setenumerate{noitemsep,topsep=0pt,parsep=0pt,partopsep=0pt}

% Titulinio aprašas
\university{Vilniaus universitetas}
\faculty{Matematikos ir informatikos fakultetas}
\department{Programų sistemų katedra}
\papertype{Programų sistemų inžinerijos laboratorinis darbas Nr. 3}
\title{Kavinės staliuko rezervavimo aplikacija}
\titleineng{Cafe table rezervation app}
\status{2 kurso 5 grupės studentai}
\author{Paulius Grigaliūnas}
\secondauthor{Karolis Staskevičius}
\thirdauthor{Modestas Dulevičius}
\fourthauthor{Albert Jurkoit}
\fifthauthor{Šarūnas Kazimieras Buteikis}
     

% \secondauthor{Vardonis Pavardonis}   % Pridėti antrą autorių
\supervisor{dr. Vytautas Valaitis}
\date{Vilnius – \the\year}

% Nustatymai
% \setmainfont{Palemonas}   % Pakeisti teksto šriftą į Palemonas (turi būti įdiegtas sistemoje)

\begin{document}
% PAKEISTA	
\maketitle
\cleardoublepage\pagenumbering{arabic}
\setcounter{page}{2}


%ANOTACIJA

\sectionnonum{ANOTACIJA}
\noindent


%TURINYS(TOC)
\tableofcontents

%ĮVADAS
\sectionnonum{Įvadas}
\noindent

\section{Verslo proceso aprašas}

\section{Išorinė analizė}

\section{Vidinė proceso analizė}

\section{Analizės rezultatai}

\section{Verslo proceso tobulinimo strategija}

\section{Sistemos naudojimo scenarijai}

\section{Įgyvendinamumo ir naudos analizė}
Šiame skyriuje bus analizuojamas sistemos įgyvendinamumas ir jos nauda.
\subsection{Operacinis įgyvendinamumas}
Inovaciniai slenksčiai:\\
Darbuotojai su žemu kompiuteriniu raštingumu gali nemokėti naudotis nauja programine įranga arba
nenorėti prisitaikyti prie pokyčių.\\
Klientai, nepratę prie naujos aplikacijos, gali nežinoti, kaip ji tiksliai veikia ir ja nesinaudoti.\\
\\
Šių inovacinių slenksčių panaikinimo būdai:\\
Darbuotojams vesti mokymus, paaiškinti apie sistemos teikiama naudą darbuotojams, geriausiai ir
greičiausiai prisitaikiusiems prie naujos aplikacijos mokėti priedus.\\
Naujai aplikacijai sukurti ir taikyti reklamos strategiją\\
Klientams sukūrti demonstracinį video įrašą, kuris supažindins su sistemos pagrindiniais panaudojimo scenarijais.

\subsection{Techninis įgyvendinamumas}
Kompiuteris su Windows OS ir bent:\newline
{\begin{itemize}
\item 2048MB RAM
\item 20GB HDD
\item 1700MHz procesoriumi
\end{itemize}

\subsection{Ekonominis įgyvendinamumas}
Reikalinga techninė įranga (serveriui):
{\begin{itemize}
\item Kompiuteris - 1000 eur
\item Nepertraukiamo maitinimo šaltinis - 180 eur
\item Tinklo įranga - 30 eur
\item Iš viso: 1210 eur
\end{itemize}
\hfill\break


Reikalinga techninė įranga (vienam restoranui):

{\begin{itemize}
\item Kompiuteris - 500 eur 
\item Nepertraukiamo maitinimo šaltinis - 180 eur
\item Iš viso: 680 eur
\end{itemize}
\break

Programinė įranga:
{\begin{itemize}
\item Operacinė sistema Windows 10 Pro - 200 eur 
\item Sistema „Covfefe“ - 10 000 eur
\item Iš viso: 10 200 eur
\end{itemize}
\hfill\break

Bendra įrangos kaina - 12 090 eur

Metinės eksplotavimo išlaidos:
{\begin{itemize}
\item Atlyginimas sistemos administratoriui - 7 200 eur
\item Interneto paslaugos - 1000 eur
\item Programinės ir techninės įrangos aptarnavimas ir atnaujinimas - 5000 eur
\item Iš viso: 13200 eur per metus
\end{itemize}
\hfill\break

Numanoma verslui atnešama nauda:\\
Registruotose restoranuose (orientuojamasi į 60 registruotų restoranų tinklą) apsilanko 250000 žmonių per mėnesį. Per metus registruotu restoranų tinklas gauna 600000
eur pelno.\\
Įdiegus sistemą “Covfefe” tikimasi padidinti staliukų užimtumą 15\% ir taip padidinti
pelną 90000 eur per metus.
\hfill\break

Sistema atsipirktų po 8 menėsių. 

\subsection{Juridinis įgyvendinamumas}
Sukuriant sistemą nebus pažeista Lietuvos respublikos konstitucija, Asmens duomenų
apsaugos įstatymas, Statistikos įstatymas ar kitų Lietuvos Respublikos teisės aktų numatyti
draudimai.\\
 Darbuotojai turi teisę į tinkamas, saugias ir sveikas darbo sąlygas. Darbuotojams
mokamas menėsinis atlyginimas, darbo valandos neviršija 8 valandų per parą. Kiekvienas
darbuotojas turi teisę į poilsį ir laisvalaikį, taip pat kasmetines mokamas atostogas.\\ 
Duomenys tvarkomi tiksliai, sąžiningai ir konfidencialiai.
\break











\section{Literatūros sąrašas}


\end{document}
