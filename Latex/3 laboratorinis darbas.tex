\documentclass{VUMIFPSkursinis}
\usepackage{algorithmicx}
\usepackage{algorithm}
\usepackage{algpseudocode}
\usepackage{amsfonts}
\usepackage{amsmath}
\usepackage{bm}
\usepackage{caption}
\usepackage{color}
\usepackage{float}
\usepackage{graphicx}
\usepackage{listings}
\usepackage{subfig}
\usepackage{wrapfig}
\usepackage{pdflscape} %Keep it to pdflscape or I can't rotate my diagram (K.S.)
\usepackage{longtable}
\usepackage[table]{xcolor}
\usepackage{multirow}
\usepackage[usestackEOL]{stackengine}
\usepackage{longtable}
\usepackage{subfig}
\usepackage{wrapfig}


\usepackage{enumitem}
%PAKEISTA, tarpai tarp sąrašo elementų
\setitemize{noitemsep,topsep=0pt,parsep=0pt,partopsep=0pt}
\setenumerate{noitemsep,topsep=0pt,parsep=0pt,partopsep=0pt}

% Titulinio aprašas
\university{Vilniaus universitetas}
\faculty{Matematikos ir informatikos fakultetas}
\department{Programų sistemų katedra}
\papertype{Programų sistemų inžinerijos laboratorinis darbas Nr. 3}
\title{Kavinės staliuko rezervavimo aplikacija}
\titleineng{Cafe table rezervation app}
\status{2 kurso 5 grupės studentai}
\author{Paulius Grigaliūnas}
\secondauthor{Karolis Staskevičius}
\thirdauthor{Modestas Dulevičius}
\fourthauthor{Albert Jurkoit}
\fifthauthor{Šarūnas Kazimieras Buteikis}
     

% \secondauthor{Vardonis Pavardonis}   % Pridėti antrą autorių
\supervisor{dr. Vytautas Valaitis}
\date{Vilnius – \the\year}

% Nustatymai
% \setmainfont{Palemonas}   % Pakeisti teksto šriftą į Palemonas (turi būti įdiegtas sistemoje)

\begin{document}
% PAKEISTA	
\maketitle
\cleardoublepage\pagenumbering{arabic}
\setcounter{page}{2}


%ANOTACIJA

\sectionnonum{ANOTACIJA}
\noindent


%TURINYS(TOC)
\tableofcontents

%ĮVADAS
\sectionnonum{Įvadas}
\noindent

\section{Verslo proceso aprašas}

\section{Išorinė analizė}

\section{Vidinė proceso analizė}

\section{Analizės rezultatai}

\section{Verslo proceso tobulinimo strategija}
Vizija:	Visi restorano maisto užsakymo ir tiekimo komponentai(užsakymai vietoje, telefonu, internetu, mobilia programėle, maisto tiekimas vietoje bei į namus) turi atnešti pelno restoranui.
Misija: Gerinti maisto tiekimo sąlygas klientams, naudojantis šiuolaikinėmis technologijomis, tobulinti mažiau pelningus komponentus.
Tikslai/Siekiai:
Didinti restorano staliukų užimtumą, lanksčiai paskirstant klientų srautus:
Gerinti maisto kokybę ieškaint naujų tiekėjų;
Sumažinti klientų aptarnavimo trukmę;
Pagerinti klientų aptarnavimo kokybę;
Didinti algas restorano darbuotojams;
Didinti klientų užimtumą laukimo metu (plėsti pramogų skaičių, tokių kaip mokami stalo žaidimai ir kt.);
Plėsti užsakymo „iš namų“ sistemą, kuriant internetinę svetainę ar mobilią programėlę, kurios leistų:
Žemėlapyje parodytų arčiausiai esančius restoranus ir jų laisvų staliukų skaičių,
Matyti restorano staliukų išplanavimą bei kurie staliukai restorane nėra užimti,
Rezervuoti staliuką restorane norimu laiku tam tikroje vietoje;

\section{Sistemos naudojimo scenarijai}

\section{Įgyvendinamumo ir naudos analizė}

\section{Literatūros sąrašas}


\end{document}
